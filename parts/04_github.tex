\begin{frame}{A brief introduction to \github (1)}
  \begin{columns}[onlytextwidth]
    \begin{column}{0.7\textwidth}
      \begin{itemize}
        \item \github, \url{https://github.com} is a website that allows people to share and collaborate on git repositories online
          \inote{open source alternatives also exist, for example GitLab. }
        \item offers users an unlimited number of public repositories to collaborate on
        \item provides a Web Interface \& Online editor for most of gits features
        \item has a few additional features in addition to repositories
      \end{itemize}
    \end{column}
    \begin{column}{0.3\textwidth}
      \includegraphics[width=0.9\textwidth]{imgs/Octocat}
    \end{column}
  \end{columns}
\end{frame}


\begin{frame}{A brief introduction to \github (2): Issue Tracking \& Milestones}
  \begin{itemize}
    \item Issues can be used to track bugs, todos and ideas for your project
    \item on github, anyone can comment on them
    \item people that have access to your repository can mark them as done
      \inote{This can even be done from within commit messages}
    \item you can use milestones to track your overall progress
  \end{itemize}
\end{frame}


\begin{frame}{A brief introduction to \github (3): Forking \& Pull Requests}
  \begin{itemize}
    \item sometimes you want to suggest specific changes in the code to a repository
    \item if you have access to the repository, you can just commit directly
    \item in other cases, you can ``fork'' the repository
      \inote{This makes a clone of the repository to your github account}
    \item you can then make changes in your fork and issue a pull request
    \item the owner of the original repository can then merge the changes back in
    \item time for a final demo
  \end{itemize}
\end{frame}